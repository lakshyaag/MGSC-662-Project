\documentclass[12pt]{article}

% Packages
\usepackage[utf8]{inputenc}
\usepackage[margin=1in]{geometry}
\usepackage{amsmath}
\usepackage{longtable}
\usepackage{booktabs}
\usepackage{hyperref}

% Document information
\title{Disaster Route Planning}
\author{Group 3 - Lakshya A., Arnav G., Om S., Nandani Y., Michael M.}
\date{\today}

\begin{document}
\maketitle
\section{Key idea}\label{key-idea}

Disasters, whether natural or man-made, often disrupt regular transit
networks and create a pressing need for efficient and adaptive routing
solutions. Our goal is to create an optimization system that can
dynamically adjust transit routes by identifying and removing affected
nodes during disasters, ensuring the continuity of essential services
and minimizing disruption.

\section{Data sources}\label{data-sources}

\subsection{GTFS feeds}\label{gtfs-feeds}

The General Transit Feed Specification (GTFS) is a standardized data
format that allows transit agencies to publish their scheduling and
routing information. This data format provides extensive details about
public transit operations, including routes, stops, schedules, and more.
We plan on downloading the GTFS data from
\href{https://www.stm.info/en/about/developers}{Société de transport de
      Montréal} and parsing through bus routes to identify the nodes.

\subsection{Disaster data}\label{disaster-data}
The \href{https://www.publicsafety.gc.ca/cnt/rsrcs/cndn-dsstr-dtbs/index-en.aspx}{Canadian
      Disaster Database (CDD)} is a comprehensive repository that archives
significant disaster events across the country. To depict the impact of
disaster, we plan to take flooding events as an example. To represent
the impacted areas during floods, we'll depict them as bounding boxes,
and disable nodes inside the bounding box.

\section{Implementation details}\label{implementation-details}

For a public transit network, it is crucial to have a robust evacuation
plan, especially for citizens without personal vehicles. We will attempt
to tackle this by optimizing evacuation routes using buses. Our model
extends the
\textbf{\href{https://en.wikipedia.org/wiki/Travelling_salesman_problem}{Travelling
            Salesman Problem}} to cater to this unique situation.

\subsection{Assumptions}\label{assumptions}

\begin{enumerate}
      \item
            The fleet size is predetermined
      \item
            There exists a single, designated drop-off location that is deemed
            safe for evacuees
      \item
            Bus stops act as both pick-up and drop-off points
\end{enumerate}

\subsection{Parameters}\label{parameters}
\begin{enumerate}
      \item
            A list of nodes rendered inaccessible due to the disaster
      \item
            Time required to travel from node $i$ to $j$
      \item
            Calculated distance between each node (based on geospatial data)
\end{enumerate}
\subsection{LP formulation}\label{lp-formulation}

\subsubsection{Decision variables}\label{decision-variables}
Binary variables $X_{ij}$ indicating if vehicle should go from node $i$ to $j$

\subsubsection{Objective function}\label{objective-function}
Minimize the travel time from the evacuation point to the safety point, ensuring evacuees are transported in the shortest possible time.
$$
      \min \quad \sum_{i}\sum_{j} T_{ij} X_{ij}
$$

\subsubsection{Constraints}\label{constraints}
\begin{enumerate}

      \item
            A limited number of buses are available for evacuation.
      \item
            The evacuation route adapts based on the disaster's impact, removing
            affected nodes from the path.
      \item
            Each pick-up spot is visited only once to prevent redundancy.
      \item
            Subtour elimination to prevent cyclic routes that don't lead to the
            destination.
\end{enumerate}

\section{Conclusion}\label{conclusion}

The above framework represents our initial insights and approach.
However, as we delve deeper into the data and understand the nuances of
the transit network, our approach may evolve.
\end{document}
